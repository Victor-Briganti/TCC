\begin{resumoutfpr}
    Com o fim da Lei de Moore e da escala de Dennard, diferentes técnicas computacionais surgiram para suprir a demanda por maior desempenho computacional e eficiência energética. A Computação Aproximada é um paradigma que visa aumentar o desempenho e a eficiência energética de aplicações em troca de pequenas perdas de precisão. A Computação Paralela também surge como um paradigma de computação que busca obter maior desempenho por meio de um melhor aproveitamento dos processadores \textit{multicore}. O \texttt{OpenMP} surge como uma ferramenta desse paradigma, facilitando o desenvolvimento de aplicações paralelas por meio de anotações de código. A proposta deste trabalho é implementar diferentes técnicas de aproximação no \texttt{OpenMP} por meio de modificações em sua infraestrutura no LLVM. As técnicas propostas são: perfuração de laços, memoização aproximada, descarte de tarefas e relaxamento de ponto flutuante. As técnicas propostas foram avaliadas utilizando um \textit{benchmark suite} composto pelas aplicações \textit{K-Means}, \textit{2MM}, \textit{Correlação}, \textit{Deriche}, \textit{Jacobi 2D}, \textit{Mandelbrot} e \textit{PI de Monte Carlo}.
\end{resumoutfpr}

%De acordo com a NBR 6028:2021, a apresentação gráfica deve seguir o padrão do documento no qual o resumo está inserido. Para definição das palavras-chave (e suas correspondentes em inglês no abstract) consultar em Termo tópico do Catálogo de Autoridades da Biblioteca Nacional, disponível em: http://acervo.bn.gov.br/sophia_web/autoridade
\begin{resumoutfpr}
    Com o fim da Lei de Moore e da escala de Dennard, diferentes técnicas computacionais surgiram para suprir a demanda por maior desempenho computacional e eficiência energética. A Computação Aproximada visa aumentar o desempenho e a eficiência energética de aplicações em troca de perda de precisão. A Computação Paralela busca obter maior desempenho por meio de um melhor aproveitamento dos processadores \textit{multicore}. O \texttt{OpenMP} é uma ferramenta que facilita o desenvolvimento de aplicações paralelas por meio de anotações de código. O objetivo deste trabalho é implementar diferentes técnicas de aproximação no \texttt{OpenMP} por meio de modificações em sua infraestrutura no LLVM. As técnicas propostas são: perforação de laços, memoização aproximada, descarte de tarefas e relaxamento de ponto flutuante. As técnicas propostas serão avaliadas em um conjunto de sete aplicações tolerantes a aproximação, permitindo analisar o impacto dessas aproximações no desempenho e na precisão. Espera-se que os resultados obtidos evidenciem o potencial da combinação entre aproximação e paralelização para a otimização de aplicações.
\end{resumoutfpr}

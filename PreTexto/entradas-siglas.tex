%%%% LISTA DE ABREVIATURAS E SIGLAS 
%%
%% Relação, em ordem alfabética, das abreviaturas (representação de uma palavra por meio de alguma(s) de sua(s) sílaba(s) ou
%% letra(s)), siglas (conjunto de letras iniciais dos vocábulos e/ou números que representa um determinado nome) e acrônimos
%% (conjunto de letras iniciais dos vocábulos e/ou números que representa um determinado nome, formando uma palavra pronunciável).
%%
%% Este arquivo para definição de abreviaturas, siglas e acrônimos é utilizado com a opção "glossaries" (pacote)

%% Abreviaturas: \abreviatura{rótulo}{representação}{definição}

\abreviatura{art.}{art.}{Artigo}
\abreviatura{cap.}{cap.}{Capítulo}
\abreviatura{sec.}{sec.}{Seção}

%% Siglas: \sigla{rótulo}{representação}{definição}

\sigla{ca}{CA}{Computação Aproximada}
\sigla{simd}{SIMD}{\textit{Single Instruction Multiple Data}}
\sigla{gpu}{GPU}{\textit{Graphics Processing Unit}}
\sigla{api}{API}{\textit{Application Programming Interface}}
\sigla{abnt}{ABNT}{Associação Brasileira de Normas Técnicas}
\sigla{cnpq}{CNPq}{Conselho Nacional de Desenvolvimento Científico e Tecnológico}
\sigla{eps}{EPS}{\textit{Encapsulated PostScript}}
\sigla{pdf}{PDF}{Formato de Documento Portátil, do inglês \textit{Portable Document Format}}
\sigla{ps}{PS}{\textit{PostScript}}
\sigla{utfpr}{UTFPR}{Universidade Tecnológica Federal do Paraná}

%% LEIA:

%% Para usar o \gls, você deve colocar a sigla aqui em \sigla

%% ADICIONAR SIGLAS: Quando você inclui alguma sigla, pode ser necessário compilar umas duas vezes para essa aparecer na lista de siglas.

%% ATENÇÃO REMOVER SIGLAS: se você remover a sigla do seu texto (não for usar mais), você deve comentar essa aqui e remover os \gls{} dessa sigla (se não vai ficar aparecendo a sigla na lista). Em caso de ERRO, quando o LaTeX informa que você ainda tem a sigla no texto, mesmo que não tenha. Você deve limpar o cache - no OverLeaf, clique no erro, vá:
%  ->view error
%%   ->(role para baixo, até o final)
%%     ->e clique em Clear cached files


%% Acrônimos: \acronimo{rótulo}{representação}{definição}
%\acronimo{gimp}{Gimp}{Programa de Manipulação de Imagem GNU, do inglês \textit{GNU Image Manipulation Program}}

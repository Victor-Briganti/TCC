%%%% ABSTRACT
%%
%% Versão do resumo para idioma de divulgação internacional.

\begin{abstractutfpr}%% Ambiente abstractutfpr
    With the end of the Moore Law and the Dennard Scaling, different computational techniques emerged to supply the demmand for high performance compution e energetic efficiency. Approximated Computing is one paradigm that tries to increase the performance and the energetic efficiency in exchange for small loses in the accuracy. Parallel Computing also emerged as a paradigm in computing that tries to bring a higher performance by the use of multicore processors. \texttt{OpenMP} is a tool that facilities the development of parallel applications by the use of code annotations. The proposal of this work is to implement different approximattion techniques inside the \texttt{OpenMP} by modifing the infraestrutura of LLVM. The techniques here proposed area: loop perforation, approximated memoization, task dropping and floating point relaxation. The techniques here proposed were evaluated using a benchmark suite composed by K-Means, 2MM, Correlation, Dercihe, Jacobi 2D, Mandelbrot e PI de Monte Carlo.
\end{abstractutfpr}

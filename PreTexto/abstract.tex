\begin{abstractutfpr}%% Ambiente abstractutfpr
    With the end of Moore's Law and Dennard scaling, different computational techniques have emerged to meet the demand for higher computational performance and energy efficiency. Approximate Computing aims to increase the performance and energy efficiency of applications in exchange for a loss of precision. Parallel Computing seeks to achieve higher performance through better utilization of multicore processors. \texttt{OpenMP} is a tool that facilitates the development of parallel applications through code annotations. The goal of this work is to implement different approximation techniques in \texttt{OpenMP} by modifying its infrastructure in LLVM. The proposed techniques are: loop perforation, approximate memoization, task dropping, and floating-point relaxation. The proposed techniques will be evaluated on a set of seven approximation-tolerant applications, allowing an analysis of the impact of these approximations on performance and accuracy. The results are expected to demonstrate the potential of combining approximation and parallelization for application optimization.
\end{abstractutfpr}

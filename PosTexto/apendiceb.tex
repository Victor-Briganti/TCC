%%%% APÊNDICE B
%%
%% Texto ou documento elaborado pelo autor, a fim de complementar sua argumentação, sem prejuízo da unidade nuclear do trabalho.

%% Título e rótulo de apêndice (rótulos não devem conter caracteres especiais, acentuados ou cedilha)
% \chapter{Roteiro da entrevista}\label{cap:apendiceb}

% \begin{table}[htb]%% Ambiente table
% \caption{Orçamento dos materiais n.\textsuperscript{o} 1.}%% Legenda
% \label{tab:tab3}%% Rótulo
% \begin{tabularx}{\textwidth}{@{\extracolsep{\fill}}lrrr}%% Ambiente tabularx
% \toprule
% Material              & \multicolumn{1}{c}{Valor (R\$)} & \multicolumn{1}{c}{Quantidade}  & \multicolumn{1}{c}{Total (R\$)} \\ \midrule
% Bomba centrífuga      & 2500,00                         & 01                              & 2500,00                         \\
% Compressor rotativo   & 3000,00                         & 01                              & 3000,00                         \\
% Manômetro diferencial & 450,00                          & 02                              & 900,00                          \\
% Termopar              & 370,00                          & 02                              & 740,00                          \\
% Válvula de esfera     & 43,00                           & 02                              & 86,00                           \\
% Tubulação de PVC      & 10,00                           & 05                              & 50,00                           \\
% Conexão de PVC        & 5,00                            & 10                              & 50,00                           \\ \midrule
%                       &                                 & \multicolumn{1}{r}{Total (R\$)} & 7326,00                         \\ \bottomrule
% \end{tabularx}
% \fonte{}%% Fonte
% \end{table}

\chapter{Considerações Finais}\label{cap:conclusao}

Este trabalho tem como objetivo melhorar o desempenho de aplicações de \gls{hpc} por meio da aplicação combinada de técnicas de aproximação e paralelização. A busca por maior desempenho e eficiência energética surge devido às limitações enfrentadas na evolução do \textit{hardware}, em decorrência do fim da Lei de Moore e da escala de Dennard.

Diante desse cenário, busca-se utilizar o paradigma da \gls{ca} como uma alternativa para ampliar o desempenho e reduzir o consumo de recursos. Aplicações que não necessitam de resultados exatos e que demandam alta carga computacional podem se beneficiar de técnicas como perfuração de laço, memoização aproximada, descarte de tarefas e relaxamento de ponto flutuante. Essas abordagens permitem ganhos expressivos de desempenho com impacto limitado na precisão, evitando modificações profundas no código-fonte.

Este trabalho propõe a criação de uma extensão para a ferramenta \texttt{OpenMP} dentro do compilador LLVM, de modo que esta passe a oferecer suporte nativo a diferentes técnicas de aproximação, integradas às diretivas de paralelização existentes. Com isso, pretende-se prover aos desenvolvedores uma interface simples e unificada para aplicar otimizações que resultem em maior desempenho, sem demandar grandes alterações na aplicação original.

Espera-se que, com a implementação dessas extensões, seja possível avaliar o impacto das técnicas aproximadas em aplicações reais, mensurando os ganhos de desempenho, as perdas de qualidade dos resultados e o custo de integração ao ambiente de execução. Para tal, será realizada a construção e análise de um conjunto de \textit{benchmarks}, permitindo caracterizar os padrões de execução, a adequação de cada técnica e o equilíbrio entre desempenho e precisão.